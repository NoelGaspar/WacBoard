\newcommand\blfootnote[1]{%
  \begingroup
  \renewcommand\thefootnote{}\footnote{#1}%
  \addtocounter{footnote}{-1}%
  \endgroup
}

\chapter{INTRODUCCIÓN}
\label{chapter:introduction}
\section{Experimentos de física de partículas}
\blfootnote{Esta memoria fue financiada por la Comisión Nacional para la Investigación Científica y Tecnológica (CONICYT) de Chile, en virtud de la concesión del fondo FONDECYT  11110165.} 


	La física de partículas es una rama de la física que estudia la naturaleza de las partículas fundamentales que constituyen la materia y la radiación \citep{wiki}. El estudio de las partículas elementales ha permitido a la humanidad responder algunas de las preguntas más profundas de la físicas, lo cual ha impactado en nuestras vidas a diferentes escalas, desde avances en la comprensión del universo, la composición de la materia y la existencia de nuevas dimensiones, hasta aplicaciones en el desarrollo de tecnologías de uso cotidiano. 

	La principal herramienta utilizada en la física de partículas son los aceleradores de partículas. El principio básico de un acelerador consiste en concentrar partículas subatómicas en un haz y luego acelerarlas a velocidades cercanas a la velocidad de la luz por medio de campos electromagnéticos. Posteriormente, estos haces se hacen colisionar dentro de un detector, ya sea contra objetivos estacionarios u otro haz de partículas viajando en la dirección opuesta.  Como resultados de estas colisiones, se liberan subproductos que son dispersados desde el punto de la colisión. Los detectores se encuentran compuestos por cientos de sensores que permiten medir distintos fenómenos a partir de las colisiones. El estudio y posterior procesamiento de los datos recopilados  provee a los científicos información sobre la naturaleza de las partículas elementales.

	Para realizar las colisiones entre partículas, es necesario alcanzar altos niveles de energía cinética. Debido a que los aceleradores deben trabajar a estas grandes escalas de energía, la física de partículas también recibe el nombre de "física de altas energías".  La nueva generación de instrumentos pretende trabajar en escalas de energía nunca antes utilizadas, denominada "escala tera" (1TeV), la cual permitirá replicar niveles de energía solo alcanzados en los orígenes del universo. En la actualidad se encuentran en curso dos proyectos que pretenden abrir las puertas a un nuevo mundo de investigación denominado el universo cuántico \citep{ilc101}: \textit{El Gran Colisionador de Hadrones} (LHC; Large Hadron Collider), desarrollado por la Organización Europea de Investigación Nuclear (CERN) actualmente en etapa de actualización. Y el \textit{Colisionador Lineal Internacional} (ILC; International Lineal Collider), un colisionador aun en desarrollo, pensado para estar operativo entre el 2010 - 2020, a cargo de esfuerzos conjuntos de diversos países coordinados mediante el \textit{Global Design Effort} (GDE) y el \textit{World Wide Study}.
 Se espera que la combinación de ambos aceleradores conducirán a grandes descubrimientos para la humanidad, como fue la confirmación de la existencia del campo de Higgs en esta década gracias al LHC, considerado uno de los logros más grandes de la ciencia del último medio siglo al contribuir a completar el modelo estándar.

	Con niveles de energía cada vez más altos y sistemas de detección cada vez más precisos, la construcción de un acelerador representa un desafío de elevada complejidad. Para alcanzar dichos objetivos, son necesarias construcciones de algunas decenas de kilómetros de radio y esfuerzos colaborativos internacionales que involucran a cientos de científicos e ingenieros a lo largo de todo el mundo. Es por esta razón que los aceleradores son considerados como la obras de ingeniería más grandes y ambiciosas jamás construidas por la humanidad.

	A su vez, la elevada complejidad que implica el desarrollo de un acelerador de partículas,  demanda el desarrollo de tecnologías en el estado del arte y representa una enorme y continua fuente de retos y desafíos, la cual a llamado la atención de la comunidad científica durante las ultimas décadas.  Los conocimientos obtenidos en la búsqueda de soluciones para estos desafíos, han ayudado a gran y pequeña escala en mejorar la calidad de vida de los humanos. Ejemplos concretos son la creación del internet, la cual fue desarrollada por el CERN como un proyecto para satisfacer la necesidad de compartir grandes cantidades de información entre distintos centros de investigación. O la creación de los syncotrones, los cuales utilizan un principio muy similar a los aceleradores para generar radiación y son utilizados en la medicina moderna tanto para estudiar enfermedades como para desarrollar medicamentos para combatirlas \citep{tuttle101}.

Esta tesis de pregrado trata sobre el diseño y la implementación de una plataforma de pruebas para el estudio y  la caracterización de un circuito integrado destinado a implementar los resultados expuestos en \citep{diegothesis} que forman parte de el trabajo colaborativo para desarrollar nuevas tecnologías que permitan mejorar el desempeño de la obtención de datos en los detectores, disminuyendo los efectos del ruido presente en la electrónica de \textit{front-end}.

\section{Electrónica en detectores de partículas}

	 Un detector en física de partículas es un sistema dedicado a recolectar las pistas sobre la identidad de las partículas (como su masa, velocidad, carga entre otras propiedades) a partir de los distintos fenómenos sensables posterior a una colisión. Comúnmente son tres los tipos de detectores utilizados en física de partícula: Los detectores de trayectoria, los cuales se encargan de reconstruir la trayectoria que dejan las partículas luego de una colisión (principalmente partículas que reaccionan poco con la materia como los muones o los neutrinos); los calorímetros, los cuales se encargan de medir la cantidad de energía al detener y absorber la energía de las partículas; y los detectores identificadores de partículas, los cuales se encargan de determinar la identidad de las partículas midiendo alguna propiedad específica (como la radiación de Cherenkov) o infiriéndola de balances respecto a otras mediciones \citep{cern101}.

	En la actualidad los detectores de partículas presentes en los colisionadores representan  grandes cámaras, comúnmente de forma cilíndrica coaxiales al eje de movimiento del haz de partículas, en donde se llevan a cabo las colisiones. Estos detectores de partículas se encuentran compuestos por capas de subdetectores, en donde cada capa tiene por objetivo obtener información de algún fenómeno específico. Cada una de estas capas se encuentra altamente segmentada por múltiples detectores que forman una unidad básica de detección (o pixel), permitiendo una mejor resolución espacial de las mediciones. Cada unidades básicas de detector en conjunto con su respectiva electrónica (o \textit{front-end} del detector), conforman un canal. La estructura general de un canal esta constituida por: un detector, un pre amplificador, un filtro, un conversor análogo digital \citep{spieler2005semiconductor}. La figura XX muestra un esquema altamente simplificado de un canal de un detector.

	La necesidad de generar continuamente sistemas de detección más complejos con mayor desempeño y mayor precisión, han motivado a grandes equipos de colaboración internacional para desarrollar nuevas tecnologías cada vez más avanzadas que permitan alcanzar los estándares requeridos. En el campo de la electrónica se desarrollan continuos esfuerzos para mejorar el desempeño de los sistemas involucrados en la detección: Reducción de componentes, ahorro de energía, minimización de ruido, aumento de ancho de banda, son algunos de los aspectos que encaminan nuevas investigaciones.
	
	
%%hasta aca vamos bien
%%OJOOOOO REVISAR EL SIGUIENTE PARRAFO!!
	En este contexto, la minimización del ruido representa en la actualidad uno de los limites fundamentales para la resolución de las mediciones. Diversas investigaciones se han llevado a cabo para contribuir al entendimiento y el desarrollo de una estrategia de mitigación de este problema. Uno de los resultados que motiva el desarrollo de la presente tesis de pregrado fue presentado en \citep{avila101}. En este, se propone una nueva metodología de análisis de ruido  en el dominio discreto del tiempo para el diseño de filtros discretos. Esta metodología representa una importante herramienta de diseño para la electrónica de física de partículas y el análisis de ruido en general. \footnote{Un ejemplo particular de otra aplicación de esta herramienta,  lo representa el diseño de filtros para el \textit{front-end} de lectura de un CCD, en donde el ruido de lectura es crítico en la calidad de las imágenes obtenidas \citep{guzman101}}

	Con el motivo de llevar a la práctica los resultados presentados, se propone en \citep{diegothesis} el diseño de un ASIC que implementa un prototipo para la segunda iteración de the bean, denominado the bean v2. Este integrado consiste en la implementación un \textit{front-end} de 32 canales para ser utilizado en the BeamCal un detector destinado a formar parte del ILC. Una de las principales características de The bean v2, es que implementa la etapa del filtro de la figura XX por medio de un integrador diferencial de capacitores conmutados con ganancia configurable digitalmente, permitiendo de este modo generar un filtro discreto configurable.
 

\section{Experimentos y adquisición de datos en física de partículas}

Una de las características interesantes de la física de partículas es el hecho de que las herramientas necesarias para lograr los objetivos propuestos, en la mayoría de los casos, aun no han sido creadas. Debido a que cada nuevo instrumento representa en si mismo un conjunto de desafíos únicos, es necesario desarrollar herramientas que permitan estar a la altura. De este modo, las necesidades por parte de la instrumentación de experimentos ha impulsado los limites de la ciencia y la tecnología en muchos aspectos y, a su vez, nuevos desarrollos en la instrumentación a menudo son posible gracias a los avances en la tecnología.  En este sentido, los experimentos en la física de partícula y la tecnología son dependientes el uno del otro. \citep{Attila}

La inagotable oferta de desafíos llama continuamente a  cientos de científicos a proponer y desarrollar nuevas teorías y prototipos hasta poder desarrollar en conjunto una solución adecuada. Sin embargo, la especificidad y complejidad de estas soluciones comúnmente implican un alto costo de desarrollo tanto en recursos humanos, de tiempo y  monetarios. 
Incluso el desarrollo de prototipos demanda la implementación de complejos sistemas que permitan recrear condiciones existentes en los aceleradores para someter a pruebas las soluciones propuestas.

En el contexto de la electrónica, el esquema general de los sistemas de experimentos de física de partículas consiste en un detector o sensor, una etapa de \textit{front-end}, una etapa dedicada a la adquisición de los datos, un sistema de alimentación, un bloque de \textit{triggers} o generador de impulsos para estimular el sistema y por último un sistema de control el cual comúnmente lleva a cabo las tareas de sincronización del resto de los bloques y la interfaz de comunicación. En la figura XX se presenta un diagrama simplificado de la estructura general de un sistema de adquisición de datos.

	Gracias a las tecnologías de fabricación disponible, una de las opciones más populares en la actualidad consiste en implementar las distintas etapas (o varias de ellas) en un ASCI. Estos presentan grandes ventajas ya que integran múltiples funciones en un mismo \textit{chip}, entregando la posibilidad incluso de repetir dichas funciones dentro de un mismo integrado para  lidiar con la alta segmentación actual de los detectores, procesando varios canales a la vez. Sin embargo, pese a que los ASIC poseen bajos costos cuando se producen en masa, poseen un alto costo de fabricación y altos tiempos de manufactura y diseño. 
	
	Otra alternativa la representan las FPGA o Arreglos de compuertas de campo programable (Field programmable Gate Arrays), las cuales son una opción muy adecuadas para experimentos en física de partículas,  debido a su alto desempeño, amplia versatilidad, sus capacidades de procesamiento de señales, ancho de banda y programabilidad. Debido a que poseen una gran cantidad de  compuertas lógicas programables, las FPGA permiten implementar nuevos diseños digitales por un bajo costo y en cortos tiempos, con la posibilidad de implementar casi cualquier aplicación especifica, a su vez permiten corregir cualquier error de diseño simplemente cargando un nuevo firmware en el dispositivo. \citep{1644925,6341473,4669290}

Como es común en este tipo de situaciones, la soluciones mas adecuadas contemplan una combinación lineal de ambas alternativas, de este modo, es posible obtener una solución que aproveche las ventajas de tanto de los ASI como de las FPGA. Es por eso que en el campo experimental de la física de partículas es posible encontrar sistemas híbridos, o incluso ASI que integren procesamiento digital.



\section{Contenido de esta tesis}
El capítulo 2 comienza con una introducción al proyecto que inspira el trabajo presentado en esta memoria, el desarrollo de una plataforma de pruebas para la segunda iteración de the Bean, un circuito integrado de aplicación especifica (ASIC, por sus siglas en inglés) el cual forma parte de una propuesta para el ILC. A continuación se presentan los principales desafíos de este trabajo, haciendo hincapié en cada una de las limitantes y requisitos necesarios. En el capítulo 3 se presenta un detalle del prototipo de the Bean V2, tanto sus especificaciones físicas como un detalle de sus especificaciones funcionales. Junto con lo anterior, se entrega el diseño de las principales pruebas a considerar con el fin de corroborar el correcto funcionamiento del integrado. En el capítulo 4 se presenta un detalle de la plataforma implementada, especificando como fueron abordados cada uno de los requerimientos estipulados en el capítulo 2. También se entregan los \textit{layout} finales desarrollados en el software EAGLE y la metodología utilizada para implementar una comunicación \textit{on line} con un computador. En el capitulo 5 se presentan los resultados obtenidos analizando el resultado de la plataforma desarrollada, del software implementado y los principales resultados de la implementación de la plataforma así como los resultados de las pruebas realizadas a the Bean V2. Finalmente, en el capítulo 6 se entregan las principales conclusiones de este trabajo, analizando los resultados y la contribución de este trabajo y se presentan ideas para futuras contribuciones.  