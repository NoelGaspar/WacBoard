\chapter{CONCLUSION}
\label{chapter:conclusion}
\section{Summary}

This thesis deals with the use of discrete-time filters to process the noise present in particle physics detectors front-end circuits. The main contributions of this work are: the development of a new mathematical framework for a design-oriented analysis of \mbox{discrete-time} filters in the \mbox{discrete-time} domain; and the design and implementation of a switched capacitor (SC) filter for arbitrary weighting function (WF) synthesis to be included in the Bean V2 IC.

One of the most important topics in particle physics instrumentation is finding the optimal WF for noise minimization. Although some WFs, such as the cusp \citep{radeka104}, are impossible to synthesize using continuous-time circuits, it was then interesting to determine the fundamental lower limit of noise that could be achieved through them. From then on, design efforts were focused on synthesizing the closest WFs to these theoretical optimal ones. Once introduced the discrete-time pulse shapers in the early 90's, the design efforts remained on this path, by using discrete-time filters to synthesize WFs similar to the continuous-time optimal WFs. However, this approach ignores the discrete-time nature of the pulse shaper, since taking into account this condition results in different optimal WFs, which for a variety of conditions could be very different to the continuos-time counterparts.  This problem leads to the main work presented in this thesis, a mathematical framework to calculate the noise of a typical detector front-end circuit from a \mbox{discrete-time} point of view, and thereby, a powerful tool to design the optimal discrete-time pulse shapers.

From a practical point of view, a generic filter for arbitrary weighting function synthesis is an ideal companion for the theoretical framework mentioned above. The design of this filter was presented in this work, framed on the design of the Bean V2,  an application specific integrated circuit (ASIC) planned to meet the BeamCal instrumentation needs. The use of this filter,  along with a proper characterization of the CSA and detector noise statistics, will allow to minimize the output referred noise on the BeamCal front-end circuit.

\section{Future work}

The mathematical framework presented in this work depends on a proper characterization of the CSA and detector noise statistics to find the optimal filter. Once estimated these parameters, the filter coefficients are computed offline, and then, they are updated in the circuit. An interesting future development could be to find a methodology to compute the filter coefficients online without having to characterize the CSA and detector noise statistics separately. To pursue this problem, the mathematical framework presented in this work represents a proper starting point. Also, given the abstraction of this work, it could find applications in other fields with similar circuit configurations, such as the current developments for astronomical instrumentation \citep{guzman101}.

The lessons learned during the design and simulation of the Bean filter prototype will prompt corrections, improvements, and upgrades for future revisions. Based just on simulation results, the most urgent correction consist of editing the filter OTA design to meet the settling time specification, and thus, the linearity specification.  Filter lab testing are about to start. Additional corrections, improvements, and upgrades are expected as result of this development phase.
